\documentclass[12pt]{article}
\usepackage{amsmath}
\newcommand{\myvec}[1]{\ensuremath{\begin{pmatrix}#1\end{pmatrix}}}
\newcommand{\mydet}[1]{\ensuremath{\begin{vmatrix}#1\end{vmatrix}}}
\newcommand{\solution}{\noindent \textbf{Solution: }}
\providecommand{\brak}[1]{\ensuremath{\left(#1\right)}}
\providecommand{\norm}[1]{\left\lVert#1\right\rVert}
\let\vec\mathbf

\title{coordinate geometry}
\author{N.Charan}

\begin{enumetrate}
\maketitle
\section*{10$^{th}$ Maths - Chapter 7}
This is Problem from- 4 from Exercise 7.3
\begin{enumerate}
\item Find the area of the quadrilateral whose vertices, taken in order, are (-4, -2), (-3, -5), (3, -2) and (2, 3).\\
\\Solution:\\
Join AC and divide the quadrilateral into two triangles.
\\We have two triangles, ΔABC and ΔACD.
\\Now
\\Area of ΔABC = 1/2 × [x1(y2 – y3) + x2(y3 – y1) + x3(y1 – y2)]\\
\\ 1/2 [(-4) {(-5) – (-2)} + (-3) {(-2) – (-2)} + 3 {(-2) – (-5)}]\\
\\= 1/2 (12 + 0 + 9)\\
\\= 21/2 square units\\
\\Area of ΔACD = 1/2 [(-4) {(-2) – (3)} + 3{(3) – (-2)} + 2 {(-2) – (-2)}]\\
\\= 1/2 (20 + 15 + 0)\\
\\= 35/2 square units\\
\\here, Area of quadrilateral ABCD = Area of ΔABC + Area of ΔACD\\
\\= (21/2 + 35/2) square units \\
\\= 28 square units.\\













\end{enumerate} 



\end{document}